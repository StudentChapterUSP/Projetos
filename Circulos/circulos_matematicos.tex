\documentclass{article}
\usepackage[utf8]{inputenc}
\usepackage[portuguese]{babel}
\usepackage[utf8]{inputenc}
\usepackage{placeins}

\usepackage{amsthm}

\title{Círculos Matemáticos}

\setlength{\parskip}{.3em}

\begin{document}
\maketitle

``Se quiser construir um barco, não diga as pessoas para juntarem madeira, dividir o trabalho, e dar ordens. No lugar disso, ensine-as a ensiar pelo mar vasto e sem fim'' - Antoine du Saint-Exupéry

A ideia dos círculos matemáticos não é obrigar jovens a fazerem exercícios de matemática, mas sim
fazê-las ver como matemática pode ser uma atividade prazerosa e útil. Para tanto, buscamos cerca de
10 jovens na faixa etária de, aproximadamente, de 9 a 13 anos.

O Programa Inicial de um encontro é o seguinte:

  \begin{itemize}
      \item1ª parte: discussão coletiva de uma teoria e/ou problema;
      \item2ª parte: realização individual de exercícios, com eventual ajuda.
  \end{itemize}

A intenção não é darmos uma aula, mas favorecer a interação ativa dos jovens com a matemática: damos o empurrão inicial, seja na explicação de uma teoria ou de um problema interessante, e buscamos fazer, por meio de perguntas ou de alguma conversa, que os alunos interajam. Essa primeira parte é coletiva e tem o objetivo de favorecer a troca de ideias e a interação.

Mas também é importante que cada um tenha um tempo e condições de pensar sozinho. Então, propomos uma série de problemas para cada aluno resolver individualmente, com a eventual ajuda de alguém da equipe (do \textit{Student Chapter}), na medida do necessário.

A princípio, os problemas e teorias são retiradas do livro {\textit Círculos Matemáticos, a experiência russa}, da editora do IMPA. No primeiro encontro, serão tratados de problemas diversos de lógica, geometria, e problemas numéricos, que não necessitem do conhecimento prévio de teoria particular. Isso será usado para entendermos melhor o nível dos presentes e, com base nisso, desenvolvermos o material para os próximos encontros.

Em um encontro típico, haveria a explanação de alguma teoria (por exemplo, o princípio da casa dos pombos), seguida da resolução de alguns problemas-exemplo e da resolução de problemas por parte dos alunos.



\end{document}

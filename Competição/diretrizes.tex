\documentclass{article}
\usepackage[utf8]{inputenc}
\usepackage[portuguese]{babel}
\usepackage[utf8]{inputenc}
\usepackage{placeins}

\usepackage{amsthm}

\title{Competição \textit{(pensar em nome melhor!)}}

\setlength{\parskip}{1em}

\begin{document}
\maketitle

  $\bullet$ Do que se trata:\\
  Competição entre ideias soluções de problemas que façam algum uso de matemática.
  As soluções não precisam ser completas: Procuramos primeiro ideias boas e buscaremos deixá-las completas depois.

  $\bullet$ Que tipo de problema?\\
  A princípio, não há restrições ao tipo de problema. No entanto, o problemas devem ter relevância.
  A relevância um problema leva em conta:
  \begin{enumerate}
      \item A quantidade de atingidos pelo problema;
      \item Quão afetada cada um dos atingidos é.
  \end{enumerate}

  $\bullet$ Que tipo de critério deve ser usado para a seleção?\\
  Devem ser levadas em conta a relevância do problema, a qualidade e a dificuldade da solução.
  A qualidade da solução é quantificada levando em conta:
  \begin{enumerate}
      \item Escalabilidade;
      \item Recursos necessários (apreciamos mais soluções baratas);
      \item Robustês (se o cenário do problema mudar um pouco, quão válida a solução proposta ainda é);
      \item Extensibilidade (quão modificada ela teria que ser para ser aplicada a outro contexto);
  \end{enumerate}
  A dificuldade da solução leva em conta:
  \begin{enumerate}
      \item Grau de abstração matemática;
      \item (quando aplicável) Quantidade de variáveis do problema consideradas (por exemplo, em um problema hídrico, pode-se levar em conta só a quantidade de chuva, ou a quantidade de chuva e características geológicas, ou isso mais químicos usados por agricultores, etc.);
  \end{enumerate}

Com base nos critérios acima, cada solução proposta deve receber uma pontuação (por critério) e as melhores serão selecionadas.
Buscaremos trabalhar essas soluções com os autores delas e, possívelmente, com professores do ICMC ou de algum outro instituto,
afim de, até o final do semestre, termos algo que possa ser aplicável ou submetido a algum orgão/instituto que possa aplicá-la.

  $\bullet$ Qual vai ser o prêmio?\\
  Precisamos pensar.

\end{document}
